\documentclass{book}

\title{git}
\usepackage{listings}
\usepackage[utf8]{inputenc}
\usepackage[margin=1.25in]{geometry}
\usepackage[colorlinks]{hyperref}
\usepackage{todonotes}
\usepackage{amsmath}
\usepackage{amssymb}

\author{Johnny English}

\begin{document}
\maketitle
{
\hypersetup{linkcolor=black}
\tableofcontents
}
\part{The basics}

\chapter{Introduction to git}
git is a very popular version control. it is very simple to use.
 
\chapter{git commands}
\section{list of all git commands}
\section{status}
\section{add}
\section{commit}
\section{checkout}
\section{branch}
\section{pull}
\section{fetch}
\section{remote}
\section{config}
config command helps configure aspects of the git repository.\\
git stores certain configuration to the repo in a config file.
git can be configured on different levels.
\begin{enumerate}
\item worktree - for a folder in a repo
\item local    - for the repo only
\item global   - for the user
\item system   - for all users
\end{enumerate}
These files have hierarchical order where the closest config rule will be applied. \\
$$worktree > local >  global > system $$
If however a configuration is not present in the closest config file or even if the config file 
does not exist then git will default to the config file of the next level.
for example 
if the global alise of the combination co means "checkout" but the local aliase of co is "config" then git will 
interpert co as config becase local configs overrule global ones.\par
The git config command is used to used to change the config file without the need to type in the configurations manually 
eventhough the final effect is the same in both approaches.\\
There are many configurations that can be changed for git.
\begin{lstlisting}[language=bash]
$ git config 
\end{lstlisting}




\part{advanced features}

\end{document}
