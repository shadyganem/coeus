\documentclass{book}

\title{LAMP}
\usepackage{listings}
\usepackage[utf8]{inputenc}
\usepackage[margin=1.25in]{geometry}
\usepackage[colorlinks]{hyperref}
\usepackage{todonotes}

\author{Shady Ganem}

\begin{document}
\maketitle
{
\hypersetup{linkcolor=black}
\tableofcontents
}

\part{LAMP}

\chapter{Introduction}
\section{What is LAMP}
LAMP stands for Linux Apache MySQL and PHP. This is very famous web dev 
stack. 

\section{Installing LAMP}
To install the LAMP stack you would aviously need a linux machine. 
In this toturial we will use raspberry pi os. it is debian based then this 
will apply to debian based linux distros.
first update the package manager
\begin{lstlisting}[language=bash]
$ sudo apt update & sudo apt upgrade
\end{lstlisting}
Then install apache server 
\begin{lstlisting}[language=bash]
$ sudo apt install apache2 -y 
\end{lstlisting}
Appache will get installed on your system. This process might take a few minute.\\
To make sure that the service is running. 
\begin{lstlisting}[language=bash]
$ sudo service apache2 status 
\end{lstlisting}
If the apache2 service is running this means your system will serve http requests on port 80. which is wonderful.\par
Now we will move to install PHP
in a very similar manner 
\begin{lstlisting}[language=bash]
$ sudo apt install php -y 
\end{lstlisting}

\end{document}
