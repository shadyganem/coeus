\documentclass{book}

\title{wordpress}
\usepackage[utf8]{inputenc}
\usepackage[margin=1.25in]{geometry}
\usepackage[colorlinks]{hyperref}
\usepackage{todonotes}
\usepackage{listings}
\lstset{basicstyle=\ttfamily,
  showstringspaces=false,
  commentstyle=\color{red},
  keywordstyle=\color{blue}
}
\usepackage{}

\author{Johnny English}


\def \bashversion {bash 1.2}

\begin{document}
\maketitle
{
\hypersetup{linkcolor=black}
\tableofcontents
}
\part{}
\chapter{secure the server}
The subject of security is of major importance when it comes to web development.
when deploying a new website we should assume that every transaction between the 
server and the client is compromised and must be secured. 

\section{Auto udpate}
security issues can be caused due to out dated software. we will install a tool for 
automating the process
In ssh terminal
\begin{lstlisting}[language=bash]
$ sudo apt update
$ sudo apt dist-upgrade
\end{lstlisting}
Now we will install the unattneded upgrade tool
\begin{lstlisting}[language=bash]
$ sudo dpkg-reconfigure --priority=low unattended-upgrades
\end{lstlisting}
Hit "YES"

\section{Create a user}
If you are running as root you should create a user in the sudo group.
it is considered bad practice to operate as root in linux. 

\begin{lstlisting}[language=bash]
$ adduser \username
\end{lstlisting}
Add the user to the sudo group
\begin{lstlisting}[language=bash]
$ \username -aG sudo \username
\end{lstlisting}

\section{Authentication key pair}
Logging in to the ssh server with a password is not a bad practice but it is less
secure than logging in using and authentication keys.

first lets create a directory that will hold they keys for our user.
\begin{lstlisting}[language=bash]
$ mkdir ~/.ssh && chmod 700 ~/.ssh
\end{lstlisting}





\end{document}
