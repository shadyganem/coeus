\documentclass{book}

\title{The linux Operating System}
\usepackage{listings}
\usepackage[utf8]{inputenc}
\usepackage[margin=1.25in]{geometry}
\usepackage[colorlinks]{hyperref} \usepackage{todonotes} 

\author{Shady Ganem}

\begin{document}
\maketitle
\tableofcontents
%Content goes here

\part{Linux Basics}

\chapter{About Linus and Linux Creation}
Linus Torvald developed the first version of Linux at 1991 as an operating 
system for the computers that are powered by the intel 80386 micoprocessor.
Linux was a student at the university of Hellsinky. 
Today Linux is a full-fleged operating system. \par
Linux is a unix like system but is not unix, it shares none of the code with unix.
Linux shares many ideas with unix and uses the unix API defined in the POSIX specifications.
Linux is an open source project developed over the internet by contributors from all around the world. 
Linus is the maintainer of the Linux kernel.  
Linux is free in every way. The source code is open and cab be easily downloaded.

\part{Kernelspace}

\chapter{Introduction}
How does the Linux kernel differ from other kernel?
we can devide kernels into two main designed: 
\begin{itemize}
\item Monolithic Kernel
\item Microkernel
\end{itemize} 
Linux is a monolithic kernel. Monolithic kernels are implemented entirlay 
as a single process that run in a single address space. Such kernels 
typically are stored on desk as a single static binary.\par
Linux supports dynamic loading of kernel modules.

\chapter{Components Of The Kernel}

\chapter{Kernel Source Code}

How to get the Linux Kernel source code?\\
Run in a terminal shell
\begin{lstlisting}[language=bash]
$ git clone git://git.kernel.org/pub/scm/linu/kernel/git/torvalds/linux.git
$ git pull 
\end{lstlisting}

\chapter{How To Build The Kernel}

\chapter{Linux Kernel Development}


\part{Userspace}

\chapter{tools}

\chapter{configurations}

\section{hostname}


\chapter{systemd}

\chapter{servers}

\section{web}
\subsection{LAMP}
The LAMP stands for Linux Apache MySQL PHP. It is a well know software stack for running a web server. \\
Installing Apache on linux machine.\\
On a debian based distro.
\begin{lstlisting}[language=bash]
sudo apt update
sudo apt install apache2
\end{lstlisting}

\section{ssh}
SSH servers enable logging into a machine via ssh (secure shell). ssh is a secure way to connect to a remote machine. It is wildy used for linux.\\

\subsection{openssh}
openssh is widly used as an ssh server, on linux and other OS as well.\\
Not to be confused with ssh client which is the tool to ssh server.\\
\begin{itemize}
\item First update current packages. it is always important to install latest patches.
\begin{lstlisting}[language=bash]
sudo apt-get update
\end{lstlisting}
\item check if openssh is already installed.
\begin{lstlisting}[language=bash]
ssh localhost
\end{lstlisting}
if it promptes you for a password then you have it installed. Otherwise you need to install it.\\
\item To install openssh
\begin{lstlisting}[language=bash]
sudo apt-get install openssh-server
\end{lstlisting}

\end{itemize}


\end{document}
