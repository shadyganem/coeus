\documentclass{book}

\title{The linux Operating System}
\usepackage{listings}
\usepackage[utf8]{inputenc}
\usepackage[margin=1.25in]{geometry}
\usepackage[colorlinks]{hyperref} \usepackage{todonotes} 
\usepackage{}

\author{Shady Ganem}

\begin{document}
\maketitle
\tableofcontents
%Content goes here

\part{Linux Basics}

\part{Kernelspace}

\part{Userspace}

\chapter{tools}

\chapter{systemd}

\chapter{servers}

\section{web}
\subsection{LAMP}
The LAMP stands for Linux Apache MySQL PHP. It is a well know software stack for running a web server. \\
Installing Apache on linux machine.\\
On a debian based distro.
\begin{lstlisting}[language=bash]
sudo apt update
sudo apt install apache2
\end{lstlisting}

\section{ssh}
SSH servers enable logging into a machine via ssh (secure shell). ssh is a secure way to connect to a remote machine. It is wildy used for linux.\\

\subsection{openssh}
openssh is widly used as an ssh server, on linux and other OS as well.\\
Not to be confused with ssh client which is the tool to ssh server.\\
\begin{itemize}
\item First update current packages. it is always important to install latest patches.
\begin{lstlisting}[language=bash]
sudo apt-get update
\end{lstlisting}
\item check if openssh is already installed.
\begin{lstlisting}[language=bash]
ssh localhost
\end{lstlisting}
if it promptes you for a password then you have it installed. Otherwise you need to install it.\\
\item To install openssh
\begin{lstlisting}[language=bash]
sudo apt-get install openssh-server
\end{lstlisting}

\end{itemize}


\end{document}
