\documentclass{book}

\usepackage{listings}
\usepackage[utf8]{inputenc}
%hyperref is for inserting hyperlinks 
\usepackage[colorlinks]{hyperref}
% geometry packsge for managing page margins
\usepackage[margin=1.25in]{geometry}

\usepackage{todonotes}

\title{The C Language}
\author{Shady Ganem}

\begin{document}
\maketitle
\tableofcontents
\listoftodos[TODO list]

\part{The C Language}
\chapter{Introduction}
\section{About the language}
The C programming langauge originally developed by Dennis Rechie and Ken Thompson in 1972, 
is a multi purpose language. The C language was developed for the development of the unix operating system.
\todo{finish writing this section}
\section{Hello, World!}
In order to write the famous beginer program in C on a unix based OS using gcc.\\
First create a new .c file. \\
Type in the terminal:
\begin{lstlisting}[language=bash]
touch hello.c
\end{lstlisting}
\begin{lstlisting}[language=C]
#include <stdio.h>

int main()
{
    printf("Hello, Wrold!\n");
}
\end{lstlisting}
To compile this program\\
Type in the terminal:
\begin{lstlisting}[language=bash]
gcc hello.c -o hello
\end{lstlisting}


\part{C Standart Library}

\chapter{Linux Specific Libraries}
\section{libs}
\subsection{errno}
This library defines the integer errno which holds the last error number.\\
To access this intger inlcude the header file errno.h\\
errno.h defines manny error values that can be examined to determine the error cause.\\

useage example:
\begin{lstlisting}[language=C]
#include <errno.h>

int main()
{
    if(mkdir("\home\username\newfolder") != 0)
    {
        if (errno == EEXIST)
        {
            printf("file or folder already exists\n");
        }
    }
} 

for more info see the man page for this nubmer.

\end{lstlisting}
\end{document}
