\documentclass[12pt]{book}

% --------- packages         ---------- %
%listings is for embeding code
\usepackage{listings}
\usepackage[utf8]{inputenc}
%hyperref is for inserting hyperlinks 
\usepackage[colorlinks]{hyperref}

% geometry packsge for managing page margins
\usepackage[margin=1.25in]{geometry}

%tikz this package is used for creating drawings
\usepackage{tikz}
%using bookmark to for table of content
%\usepackage{bookmark}

% ---------- front page info ---------- %
\title{How To \LaTeX}
\author{Shady Ganem}

% ---------- end of preamble ---------- %
\begin{document}
\maketitle
\section*{Preface}
\tableofcontents
\newpage
% ---------- part 1 -------------- %
\part{Basics of \LaTeX}
\chapter{The Basics}
\section{Introdcution to \LaTeX \ }
\subsection{What is \LaTeX \ }
First let's begin with the pronunciation. \LaTeX \ is pronounced Lay-tek or Lah-tek and not Lay-teks. 
That is something that I got wrong when I first crossed paths with \LaTeX \ and I am making sure you won't.\\
Now that we know how to pronounce the name corrently let's dive into what is \LaTeX. Well, \LaTeX \ is a software tool 
for creating documents. The user writes code in plain text. The code is later processed to genereate a 
a document with all the visual aspects taken care of. \LaTeX, unlike other word processing tools let's the user focus on the content and leave 
the visual asspects of the document to the software engine. \\
There are two ways to use \LaTeX, one is to downlaod packages onto your computer, 
and the second is to use a cloud based editor.\\
I use the \href{https://tug.org/texlive/}{TEX Live} Distrobution on Linux. Mac Users can use
\href{https://www.tug.org/mactex/}{MacTEX}. It is important to note that this method could 
require additional tweeking to get it to work properly.\\
Another good way to create documents with \LaTeX \ is by using a cloud based editor. 
\href{https://www.overleaf.com}{Overleaf} is a robust enough editor for students and small projects.
Ther are many great resources to learn \LaTeX \ online. \\
Regardless of which environment you choose for using \LaTeX \ you will work with two windows. 
One window is used for typing commands and content, and another window for viewing the porcessed output document.\\
\LaTeX \ generates many types of out files but the most used one is PDF (Portable Document Format)

\subsection{Syntax}
\subsection{Document Structure}
A \LaTeX \ source code file has two main parts. The preamble and the document.
The preamble does not contain any content, rather it holds information about the document. 
The first line of \LaTeX \ document is the documentclass command. 
\begin{lstlisting}[language={[LaTeX]TeX}]
\documentclass[12pt]{book}
\end{lstlisting}
This command tells the compiler the most general inforamtion, which is the type of
document to create, as well as what are it's most general properties.
The preamble can also include declaration of the packages that will be used to create the document.
\newline
The second part of the document is the document environment. An environment in \LaTeX \ is a block which will be processed aaccording to specific rules. 
\newline The document environment is the outermost container that will hold the
document. The environment block is denoted by two delimiters.
\begin{lstlisting}[language={[LaTeX]TeX}]
\begin{document}
%document content goes here
\end{document{
\end{lstlisting}
Inside the document delimiters 
The most basic document in \LaTeX \ will look like this.
\begin{lstlisting}[language={[LaTeX]TeX}]
\documentclass{article}
%preamble contonet goes here
\begin{document}
%document content goes here
\end{document{
\end{lstlisting}
\newpage
\section{Compiling \LaTeX \ code}
\subsection{Compiling on a Linux machine}
To compile on a Linux machine you need to install the necessary packages.

On a Debian based distrobution.
\begin{lstlisting}[language=bash]
sudo apt-get install texlive-full 
\end{lstlisting}
On a Fedora based distrobution. use one of the following packages.
\begin{lstlisting}[language=bash]
sudo dnf install texlive-scheme-basic
sudo dnf install texlive-scheme-medium
sudo dnf install texlive-scheme-full
\end{lstlisting}
\newpage
\subsection{Known issues}
\LaTeX \ requires double compilation for ceratin information to be properly embeded in the document.
Usually this is first noticed when trying to create a talbe of contents command
\begin{lstlisting}[language={[LaTeX]TeX}]
\documentclass{article}
\begin{document}
\tableofcontents
\chapter{Chapter 1}
\newpage
\section{Section 1}
\end{document}
\end{lstlisting}
this could be confusing for newcomers to \LaTeX \ but now you know.
\newline
\subsection{\LaTeX \ files}
\newpage
\chapter{Packages}
\section{what is a package}
\subsection{How to use a package}
\subsection{Frequently used packages}
\subsubsection{hyperref}
The hyperref pakcage is used for creating links in the document as well as links 
for sites out of the document. Such as links to websites or other apps.
\subsubsection{inputenc}
\subsubsection{listings}
\subsubsection{geometry}
\subsubsection{tikz}
\newpage
\chapter{Documents}
\section{Document style}
\subsection{Margins}
\section{Document types}
\subsection{Books}
\subsection{Articles}
\subsection{Beamers}


% ----- part 2 --------- %
\part{How To \LaTeX}
\section{Introduction}
\chapter{Computer Science Documentation}
\section{Documentaion Automation}
\begin{verbatim}
\documentclass{book}
\end{verbatim}

\newpage
How to insert a code block into a \LaTeX \ document?
\newline One way to emed code in the \LaTeX  file is by using a verbatim 
block. Verbatim blocks will embed the text as it 
appeard in the block
\newline For exmaple:
\begin{verbatim}
\begin{verbatim}
Text in the verbatim block not processed as latex code.
\begin{verbatim}
\end{verbatim}
Another way is to use lstlisting block
The lstlisting is part of the listings package.
For more documentation about the listings package goto this \href{https://ctan.math.illinois.edu/macros/latex/contrib/listings/listings.pdf}{link}
\newline For example:
\begin{verbatim}
\begin{lstlisting}[language=python]
if __name__=="__main__":
    print("hello, world!")
\end{lstlisting}
\end{verbatim}
\subsection{Using GNU make}
The GNU make tool is widly used to create \LaTeX \ documents since it is very suited for such a task. 
example code:
\begin{lstlisting}[language=make]

OUTDIR = $(shell pwd)
TEX = pdflatex -shell-escape -interaction=nonstopmode -file-line-error
PDFOUT = how_to_latex 
PDF_TARGET = $(addsuffix .pdf, $(PDFOUT)) 
LATEXIN = main.tex

all: pdf

pdf: $(PDF_TARGET)

$(PDF_TARGET): $(LATEXIN)
	$(TEX) -output-directory=$(OUTDIR) -jobname=$(PDFOUT) $(LATEXIN)
	$(TEX) -output-directory=$(OUTDIR) -jobname=$(PDFOUT) $(LATEXIN)
	echo "cleaning .aux and .log"
	rm -f $(addsuffix .log, $(PDFOUT)) $(addsuffix .aux, $(PDFOUT)) $(addsuffix .out, $(PDFOUT)) $(addsuffix .toc, $(PDFOUT))
	echo "coping to web server"
	sudo cp $(PDF_TARGET) /var/www/html/pdf/
	echo -e "\033[0;32m Target Made Successfully \033[0m"

delete web:
	sudo rm /var/www/html/pdf/$(PDF_TARGET)
	
clean:
	rm  -f *.pdf *.aux *.dvi *.log *.toc
\end{lstlisting}
\end{document}
