\documentclass[12pt]{book}

% --------- packages         ---------- %
%listings is for embeding code
\usepackage{listings}
\usepackage[utf8]{inputenc}
%hyperref is for inserting hyperlinks 
\usepackage[colorlinks]{hyperref}

% geometry packsge for managing page margins
\usepackage[legalpaper, margin=2in]{geometry}

%using bookmark to for table of content
%\usepackage{bookmark}

% ---------- front page info ---------- %
\title{How To \LaTeX}
\author{Shady Ganem}

% ---------- end of preamble ---------- %
\begin{document}
\maketitle
\section*{Preface}
\tableofcontents
\newpage
% ---------- part 1 -------------- %
\part{Basics of \LaTeX}
\chapter{The Basics}
\section{An introdcution to \LaTeX \ }
\subsection{Syntax}
\subsection{Document Structure}
A latex file has two main parts. The preamble and the document.
The preamble does not have content only information about the document. 
The first line of latex document is the documentclass. 
\begin{verbatim}
\documentclass[12pt]{book}
\end{verbatim}
This command tells the compiler the most general inforamtion, Which is type of
document to create, and what are it's properties.
The preamble can also include declaration of which packages will be used to create the document.
\newline
The second part of the document is the document itself. This is denoted by two 
\section{Compiling \LaTeX \ code}
\subsection{Compiling on a Linux machine}
To compile on a Linux machine you need to install the necessary packages.

On a Debian based distrobution.
\begin{lstlisting}[language=bash]
sudo apt-get install texlive-full 
\end{lstlisting}
On a Fedora based distrobution. use one of the following packages.
\begin{lstlisting}[language=bash]
sudo dnf install texlive-scheme-basic
sudo dnf install texlive-scheme-medium
sudo dnf install texlive-scheme-full
\end{lstlisting}
\newpage
\subsection{Known issues}
\LaTeX \ requires double compilation for ceratin information to be properly embeded in the document.
Usually this is first noticed when trying to create a talbe of contents command
\begin{verbatim}
\tableofcontents
\end{verbatim}
this could be confusing for new commers to \LaTeX \ but now you know.
\newline
\subsection{\LaTeX \ files}
\chapter{Packages}
\section{what is a package}
\subsection{How to use a package}
\subsection{Frequently used packages}
\subsubsection{hyperref}
The hyperref pakcage is used for creating links in the document as well as links 
for sites out of the document. Such as links to websites or other apps.
\subsubsection{inputenc}
\chapter{Documents}
\section{Document style}
\subsection{Margins}
\section{Document types}
\subsection{Books}
\subsection{Articles}
\subsection{Beamers}


% ----- part 2 --------- %
\part{How To \LaTeX}
\section{Introduction}
\chapter{Computer Science Documentation}
\section{Documentaion Automation}
\begin{verbatim}
\documentclass{book}
\end{verbatim}

\newpage
How to insert a code block into a latex document?
\newline One way to emed code in the \LaTeX  file is by using a verbatim 
block. Verbatim blocks will embed the text as it 
appeard in the block
\newline For exmaple:
\begin{verbatim}
\begin{verbatim}
Text in the verbatim block not processed as latex code.
\begin{verbatim}
\end{verbatim}
Another way is to use lstlisting block
The lstlisting is part of the listings package.
For more documentation about the listings package goto this \href{https://ctan.math.illinois.edu/macros/latex/contrib/listings/listings.pdf}{link}
\newline For example:
\begin{verbatim}
\begin{lstlisting}[language=python]
if __name__=="__main__":
    print("hello, world!")
\end{lstlisting}
\end{verbatim}
\subsection{Using GNU make}
\end{document}

